\documentclass{article}

\usepackage[utf8]{inputenc}
\usepackage{graphicx} % For inserting images
\usepackage{minted}   % For typesetting code
\usepackage{eulervm}  % We suggest the Euler typeface for math, but feel free to 
\usepackage{charter}  % This typeface works well with Euler
\usepackage{xcolor}

\usepackage{amsmath}
\usepackage{amssymb}
\usepackage{float}
\usepackage{bm}

\usepackage[T1]{fontenc}       % Use T1 encoding

\title{Assignment 3}

% Single author assignments
\author{Arturo Abril Martinez (2813498)}

% Group  assignments
% \author{FirstName1 LastName2 (Student Number) \\ FirstName2 LastName2 (Student number)\\ FirstName3 LastName3 (Student number)}

\date{December 2024}


% blackboard
\newcommand{\R}{{\mathbb R}}
\newcommand{\N}{{\mathbb N}}
\newcommand{\E}{{\mathbb E}}

% bold for vectors and matrices
\newcommand{\U}{{\mathbf U}}
\newcommand{\V}{{\mathbf V}}
\newcommand{\W}{{\mathbf W}}
\newcommand{\bu}{{\mathbf u}}
\newcommand{\bv}{{\mathbf v}}
\newcommand{\bw}{{\mathbf w}}
\newcommand{\x}{{\mathbf x}}
\newcommand{\y}{{\mathbf y}}
\newcommand{\bt}{{\mathbf t}}
\newcommand{\z}{{\mathbf z}}
\newcommand{\X}{{\mathbf X}}
\newcommand{\Y}{{\mathbf Y}}
\newcommand{\Z}{{\mathbf z}}

% gray
\definecolor{kc}{HTML}{999999}

% blue 1
\definecolor{cb}{HTML}{3F6797}
\definecolor{cbl}{HTML}{4F81BD}
\definecolor{cbll}{HTML}{7BA1CD}
\definecolor{cblll}{HTML}{A7C0DE}

% blue 2
\definecolor{cc}{HTML}{006EA6}
\definecolor{ccl}{HTML}{0089CF}
\definecolor{ccll}{HTML}{40A7DB}
\definecolor{cclll}{HTML}{7FC4E7}

% green
\definecolor{cg}{HTML}{7C9647}
\definecolor{cgl}{HTML}{9BBB59}
\definecolor{cgll}{HTML}{B4CC82}
\definecolor{cglll}{HTML}{CDDDAC}

% purple
\definecolor{cp}{HTML}{665082}
\definecolor{cpl}{HTML}{8064A2}
\definecolor{cpll}{HTML}{A08BB9}
\definecolor{cplll}{HTML}{BFB1D1}

% teal
\definecolor{ct}{HTML}{3C8A9E}
\definecolor{ctl}{HTML}{4BACC6}
\definecolor{ctll}{HTML}{78C1D4}
\definecolor{ctlll}{HTML}{A5D5E3}

% orange
\definecolor{co}{HTML}{C67838}
\definecolor{col}{HTML}{F79646}
\definecolor{coll}{HTML}{F9B074}
\definecolor{colll}{HTML}{FBCBA2}

\newcommand{\kc}[1]{{\textcolor{kc}{ #1 }}}

\newcommand{\cb}[1]{{\textcolor{cb}{ #1 }}}
\newcommand{\cbl}[1]{{\textcolor{cbl}{ #1 }}}
\newcommand{\cbll}[1]{{\textcolor{cbll}{ #1 }}}
\newcommand{\cblll}[1]{{\textcolor{cblll}{ #1 }}}

\newcommand{\cc}[1]{{\textcolor{cc}{ #1 }}}
\newcommand{\ccl}[1]{{\textcolor{ccl}{ #1 }}}
\newcommand{\ccll}[1]{{\textcolor{ccll}{ #1 }}}
\newcommand{\cclll}[1]{{\textcolor{cclll}{ #1 }}}

\newcommand{\cg}[1]{{\textcolor{cg}{ #1 }}}
\newcommand{\cgl}[1]{{\textcolor{cgl}{ #1 }}}
\newcommand{\cgll}[1]{{\textcolor{cgll}{ #1 }}}
\newcommand{\cglll}[1]{{\textcolor{cglll}{ #1 }}}

\newcommand{\cp}[1]{{\textcolor{cp}{ #1 }}}
\newcommand{\cpl}[1]{{\textcolor{cpl}{ #1 }}}
\newcommand{\cpll}[1]{{\textcolor{cpll}{ #1 }}}
\newcommand{\cplll}[1]{{\textcolor{cplll}{ #1 }}}

\newcommand{\ct}[1]{{\textcolor{ct}{ #1 }}}
\newcommand{\ctl}[1]{{\textcolor{ctl}{ #1 }}}
\newcommand{\ctll}[1]{{\textcolor{ctll}{ #1 }}}
\newcommand{\ctlll}[1]{{\textcolor{ctlll}{ #1 }}}

\newcommand{\co}[1]{{\textcolor{co}{ #1 }}}
\newcommand{\col}[1]{{\textcolor{col}{ #1 }}}
\newcommand{\coll}[1]{{\textcolor{coll}{ #1 }}}
\newcommand{\colll}[1]{{\textcolor{colll}{ #1 }}}

% partial symbol in gray
\newcommand{\kp}{{\kc{\partial}}}

\DeclareMathOperator*{\argmin}{arg\,min}
\DeclareMathOperator*{\argmax}{arg\,max} 

\begin{document}

\maketitle

\section{Answers}

\paragraph{question 1} The learnable weights of the conv layer are represented as a tensor of shape:

\medskip

\noindent \mintinline{Python}|(output_channels, input_channels, kernel_height, kernel_width)|

\medskip
\noindent This is the set of \emph{filters} of the convolutional layer. More explicitly (and excluding the bias terms) we will have \mintinline{Python}|W0, W1, W2, ...| and so on up until the number given by \mintinline{Python}|output_channels|. The total number of parameters will then be: 

\medskip
\noindent \mintinline{Python}|kernel_width * kernel_height * input_channels * output_channels|. 

\medskip
\noindent The pseudocode to implement the convolution is:

\begin{minted}{Python}
Y = \
torch.zeros((batch_size, output_channels, output_height, \
    output_width))

for b in range(batch_size):
    for ch in range(output_channels)
        for row in range(output_height):
            for col in range(output_width):
                Y[b,ch,row,col] = \
                ( \
                    X[b,:,row*S:hk+row*S,col*S:wk+col*S] \
                    * W[ch,:,:,:]
                ).sum()
return Y
\end{minted}

\noindent Where \mintinline{Python}|S| represents the stride and \mintinline{Python}|hk| and \mintinline{Python}|wk| the kernel (or filter) height and width respectively. Note that the multiplication inside the \mintinline{Python}|sum()| is element-wise, and in the pseudocode \mintinline{Python}|W| is the set of all weight tensors, so that 

\noindent \mintinline{Python}|W[0,:,:,:] = W0|. \emph{Note: I assume the input images are already padded, if they need to be.}

\paragraph{question 2} Given that the size of the input is \((C_{in}, H_{in}, W_{in})\), we can calculate the size of the output as:
\[
    W_{out} = \frac{W_{in} - K + 2P}{S} + 1
\]
\(H_{out}\) can be calculated using the same formula by symmetry, changing \(W_{in} \rightarrow H_{in}\). The size of the output is then \((C_{out}, H_{out}, W_{out})\) where \(C_{out}\) can be chosen to be any number. \(K, S\) and \(P\) are the kernel size, stride and padding respectively\footnote{I have ommited the batch size in the calculation. Of course, for every tensor in the input batch we would have an output tensor, which dimensions can be calculated with the formula provided.}.

\paragraph{question 3} Below is the pseudocode for the unfold operation. \mintinline{Python}|p, k| and \mintinline{Python}|S| represent the total number of patches, number of values per patch and stride respectively. \mintinline{Python}|w| (\mintinline{Python}|wk|) and \mintinline{Python}|h| (\mintinline{Python}|hk|) are the input (kernel) width and height respectively. \emph{Note: I assume the input images are already padded, if needed to be.}

\begin{minted}{Python}
Y = torch.zeros((batch_size, p, k))
for b in range(batch_size):
    # iterate over rows in the output
    # each row will be a flattened image patch
    for row in range(p):
        # select patch
        i = row//w_out  # transition row every `w_out` patches
        j = row%(w-wk+1) # when `row` exceeds maximum, start over
        start_row, start_col = i*S, j*S
        patch = X[
            b, :, start_row : start_row+hk, \
            start_col : start_col+wk
        ]

        # flatten patch
        counter = 0
        for ch_patch in range(input_channels):
            for row_patch in range(hk):
                for col_patch in range(wk):
                    Y[b, row, counter] = \
                    patch[ch_patch, row_patch, col_patch]
                    counter += 1
# This way the output will have size (batch_size, p, k)
return Y
\end{minted}

\paragraph{question 4} To calculate the backward of the con2d operation, we first present some notation. We first present all the tensors involved and what each of their indices represent. Later, we will use dummy indices for the caclculations, always keeping in mind what each of them represent, which will be determined by their postion. First, the constants:

\begin{itemize}
    \item \(b, c, h, w\): Batch size, input channels, input width and input height.
    \item \( h_k, w_k\): Kernel height and kernel width.
    \item \(c_{out}, h_{out}, w_{out}\): Output channels, output height and output width.
    \item \(p\): Total number of patches. Resulting from doing \(h_{out} * w_{out}\).
    \item \(k\): Total number of values per patch. Resulting from doing \(c * h_k * w_k\)
\end{itemize}

Now we define the tensors involved. The indices are now named with what they represent. We will later have to remember this:
\begin{itemize}
    \item Input batch: \(\mathbf{X} = X_{bchw}\)
    \item Unfolded input: \(\mathbf{U} = U_{bkp}\)
    \item Unfolded input transposed: \(\mathbf{\tilde{U}} = \tilde{U}_{bpk} = \textbf{U}.\text{transpose}(1, 2)\)
    \item Kernel matrix: \(\mathbf{W} = W_{c_{out}ch_kw_k}\)
    \item Kernel matrix reshaped and transposed: \(\mathbf{\tilde{W}} = \tilde{W}_{kc_{out}} = \)
    
    \( = \mathbf{W}.\text{reshape}(c_{out}, c * h_k * w_k).\text{transpose}()\)
    \item Output without reshaping: \(\mathbf{Y}' = Y'_{bpc_{out}} = \sum_k \tilde{U}_{bpk}\tilde{W}_{kc_{out}}\)
    \item Output: \(\mathbf{Y} = Y_{bc_{out}h_{out}w_{out}} = \)
    
    \( = \mathbf{Y}'.\text{transpose}(1, 2).\text{reshape}(b, c_{out}, h_{out}, w_{out})\)
\end{itemize} 

Before calculating \(\mathbf{W}^\nabla\) and \(\mathbf{X}^\nabla\) we present some useful derivatives. Note that there will be matrices which elements have a one to one correspondence, but their shapes are different. This is the case for example of \(\mathbf{Y}\) and \(\mathbf{Y}'\). When doing derivatives of one w.r.t. the other, we will have simple deltas for the dimensions that match, and ``special'' deltas for those dimensions that not match, but we still have a one to one correspondence of elements. This will become clear in the following derivatives.
\begin{align*}
    \frac{\partial Y_{abcd}}{\partial Y'_{efg}} = \delta_{ae}\delta_{bg}\delta_{cd, f} ~~;~~ \frac{\partial Y'_{efg}}{\partial \tilde{W}_{hi}} = \tilde{U}_{efh}\delta_{gi} ~~;~~ \frac{\partial \tilde{W}_{hi}}{\partial W_{abcd}} = \delta_{ia}\delta_{h, bcd}
\end{align*}

The deltas of the type \(\delta_{ab,c}\) appear when we are dealing with different shapes but there is still a one to one correspondence of elements. In this case we want to pick the element \(ab\) that corresponds to \(c\)\footnote{If this is not still clear, imagine that we have a tensor \(\mathbf{X}\) with elements \(X_{11}, X_{12}, X_{21}\) and \(X_{22}\). We can ``flatten'' this into a vector of four values \(\mathbf{x}\). \(X_{11}\) corresponds to \(x_1\), \(X_{12}\) to \(x_2\) and so on. The derivative \(\partial X_{ab} / \partial x_c = \delta_{ab, c}\).}.

\pagebreak % Add a new page so the codeblock isn't broken up.

\appendix
\section{Appendix}

\end{document}
